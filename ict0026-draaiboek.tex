\documentclass[10pt,a4paper]{report}

\let\Chaptermark\chaptermark
\def\chaptermark#1{\def\Chaptername{#1}\Chaptermark{#1}}
\let\Sectionmark\sectionmark
\def\sectionmark#1{\def\Sectionname{#1}\Sectionmark{#1}}
\let\Subsectionmark\subsectionmark
\def\subsectionmark#1{\def\Subsectionname{#1}\Subsectionmark{#1}}
\let\Subsubsectionmark\subsubsectionmark
\def\subsubsectionmark#1{\def\Subsubsectionname{#1}\Subsubsectionmark{#1}}

\input{ids-parameters}

\usepackage{xparse}
\usepackage{array}
\usepackage{colortbl}
\usepackage{textcomp}
\usepackage{verbatim}
\usepackage[dvipsnames]{xcolor}
\usepackage{fancyvrb}
\usepackage{longtable}

% redefine \VerbatimInput
\RecustomVerbatimCommand{\VerbatimInput}{VerbatimInput}%
{fontsize=\small,
 %
 frame=single,  % top and bottom rule only
 framesep=1em, % separation between frame and text
 samepage=true,
 rulecolor=\color{Gray},
 fillcolor=\color{Yellow},
 %
% commandchars=\|\(\), % escape character and argument delimiters for
                      % commands within the verbatim
 commentchar=*        % comment character
}

%Function to read in text to table
\ExplSyntaxOn
\NewDocumentCommand{\addcolumn}{mm}
 {
  % #1 is the file to read, #2 the list of items to add, separated by \\
  \fabian_addcolumn:nn { #1 } { #2 }
 }

% some variables
\seq_new:N \l__fabian_additions_seq
\tl_new:N \l__fabian_table_body_tl
\tl_new:N \l__fabian_item_tl
\ior_new:N \g_fabian_import_stream

\cs_new_protected:Npn \fabian_addcolumn:nn #1 #2
 {
  % clear the variable containing the table body
  \tl_clear:N \l__fabian_table_body_tl
  % create a sequence from the second argument
  \seq_set_split:Nnn \l__fabian_additions_seq { \\ } { #2 }
  % start reading the file
  \ior_open:Nn \g_fabian_import_stream { #1 }
  % at each line ...
  \ior_map_inline:Nn \g_fabian_import_stream
   {
    % 1. detach the leftmost item from the sequence
    \seq_pop_left:NN \l__fabian_additions_seq \l__fabian_item_tl
    \tl_put_right:NV \l__fabian_table_body_tl \l__fabian_item_tl
    % 2. add the line and a trailing &
    \tl_put_right:Nn \l__fabian_table_body_tl { & ##1 }
    % 4. add the trailing \\
    \tl_put_right:Nn \l__fabian_table_body_tl { \\ \hline }
   }
  % deliver the result
  \tl_use:N \l__fabian_table_body_tl
 }

\ExplSyntaxOff

%Function to read in text to table
\ExplSyntaxOn
\NewDocumentCommand{\addcolumnfile}{mm}
 {
  % #1 is the file to read
  \peter_addcolumnfile:nn { #1 }
 }

% some variables
\seq_new:N \l__peter_additions_seq
\tl_new:N \l__peter_table_body_tl
\tl_new:N \l__peter_item_tl
\ior_new:N \g_peter_import_stream

\cs_new_protected:Npn \peter_addcolumnfile:nn #1
 {
  % clear the variable containing the table body
  \tl_clear:N \l__peter_table_body_tl
  % start reading the file
  \ior_open:Nn \g_peter_import_stream { #1 }
  % at each line ...
  \ior_map_inline:Nn \g_peter_import_stream
   {
    % 1. add the line and the trailing \\
    \tl_put_right:Nn \l__peter_table_body_tl { ##1 \\ \hline }
   }
  % deliver the result
  \tl_use:N \l__peter_table_body_tl
 }

\ExplSyntaxOff

\usepackage[dutch]{babel}
\usepackage{colortbl}

\usepackage{graphicx}
\DeclareGraphicsExtensions{.pdf,.png,.jpg}

\usepackage{geometry}
\geometry{a4paper,total={170mm,247mm},left=20mm,top=20mm}

% Define header and footer for normal pages
\usepackage{fancyhdr}
\setlength{\headheight}{36pt}
\pagestyle{fancyplain}
\pagenumbering{roman}
\lhead{\includegraphics[scale=0.8]{images/ProrailLogo}}
\rhead{Draaiboek \DienstChange \ \Dienst \ \DienstVersie \\Versie \DraaiboekVersie\\}
\chead{\makebox[\linewidth]{\rule{\textwidth}{0.4pt}}}
\rfoot{\ \\Pagina \thepage}
\lfoot{\ \\\DienstChange \ \Dienst \ \DienstVersie \ (\DraaiboekVersie-\DraaiboekStatus), \today \\\Copyright}
\cfoot{\makebox[\linewidth]{\rule{\textwidth}{0.4pt}}}

%% Use no serif font for text, courier for commands
\newcommand*{\myfont}{\fontfamily{lmss}\selectfont}
\newcommand*{\monofont}{\fontfamily{pcr}\selectfont}

\setlength{\parindent}{0pt}

\begin{document}

\myfont

%Title Page
\title{
\includegraphics[scale=0.2]{images/ProrailKopLogo}
\makebox[\linewidth]{\rule{\textwidth}{0.4pt}}
\vfill
\DienstChange \ \Dienst \ \DienstVersie \ \DienstOmgeving\\
\ \\
Draaiboek\\
\ \\
DBK
\makebox[\linewidth]{\rule{\textwidth}{0.4pt}}
\vfill
\small
\author{ICT Operations}
\begin{tabular}{| l | l |}
\hline
\cellcolor[gray]{0.84}Kenmerk & \DraaiboekOmschrijving\\
\hline
\cellcolor[gray]{0.84}Auteur & \DraaiboekAuteur\\
\hline
\cellcolor[gray]{0.84}Eigenaar & \DienstEigenaar\\
\hline
\cellcolor[gray]{0.84}Versie & \DraaiboekVersie\\
\hline
\cellcolor[gray]{0.84}Versiedatum & \today\\
\hline
\cellcolor[gray]{0.84}Status & \DraaiboekStatus\\
\hline
\multicolumn{2}{|l|}{\large \cellcolor[gray]{0.84} \Copyright} \\
\hline
\end{tabular}
}

\maketitle

\ 

\Large{\textbf{Versiebeheer}}
\\

\large
\begin{tabular}{| l | l | p{8cm} | l |}
\hline
\rowcolor[gray]{0.84}Versie & Datum & Omschrijving & Auteur\\
\hline
\addcolumnfile{versiebeheer.txt}
\hline
\DraaiboekVersie & \today & \DraaiboekOmschrijving & \DraaiboekAuteur \\
\hline
\end{tabular}

\ 

\Large{\textbf{Distributielijst}}
\\

\large
\begin{tabular}{| l | l | l | l |}
\hline
\rowcolor[gray]{0.84}Versie & Naam & Functie & Bedrijf\\
\hline
\addcolumnfile{distributie.txt}
\hline
\end{tabular}
\\

\Large{\textbf{Goedkeuring}}
\\

\large
\begin{tabular}{| l | l |}
\hline
\cellcolor[gray]{0.84}Bedrijf & \GoedkeuringBedrijf\\
\hline
\cellcolor[gray]{0.84}Naam & \GoedkeuringNaam\\
\hline
\cellcolor[gray]{0.84}Functie & \GoedkeuringFunctie\\
\hline
\cellcolor[gray]{0.84}Paraaf & \\
\cellcolor[gray]{0.84} \  & \\
\cellcolor[gray]{0.84} \  & \\
\cellcolor[gray]{0.84} \  & \\
\hline
\cellcolor[gray]{0.84}Datum & \ \\
\hline
\end{tabular}

\tableofcontents
\newpage
\pagenumbering{arabic}
\chapter{Scope}
\section{Identificatie}
\begin{tabular}{| l | l |}
\hline
\cellcolor[gray]{0.84}Kenmerk & \DienstChange \ \Dienst \ \DienstVersie\\
\hline
\cellcolor[gray]{0.84}Versie & \DraaiboekVersie\\
\hline
\cellcolor[gray]{0.84}Titel & \DraaiboekOmschrijving\\
\hline
\end{tabular}
\section{Systeemoverzicht}
\input{systeemoverzicht.txt}

\section{Documentoverzicht}
Hoofdstuk 1 geeft een korte inleiding tot het systeem en de inhoud van dit document.

Hoofdstuk 2 bevat een overzicht van gerefereerde documenten.

Hoofdstuk 3 beschrijft de benodigdheden.

Hoofdstuk 4 beschrijft de uit te voeren werkzaamheden.

Hoofdstuk 5 beschrijft het fallback scenario.

Hoofdstuk 6 bevat het communicatieplan.
\section{Afkortingen en begrippen}
\begin{tabular}{| l | l |}
\hline
\rowcolor[gray]{0.84}Afkorting & Omschrijving\\
\hline
\addcolumnfile{afkortingen.txt}
\hline
\end{tabular}
 \\
 \\
 
\begin{tabular}{| l | l |}
\hline
\rowcolor[gray]{0.84}Begrip & Omschrijving\\
\hline
\addcolumnfile{begrippen.txt}
\hline
\end{tabular}
\chapter{Referentiedocumenten}
In dit document wordt gerefereerd aan de volgende documentatie:
 \\

\begin{tabular}{| l | l | l | l |}
\hline
\rowcolor[gray]{0.84}Kenmerk & Versie & Titel & Auteur\\
\addcolumnfile{referentie.txt}
\hline
\end{tabular}
\chapter{Benodigdheden}
\section{Locatie}
\begin{tabular}{| l | l | l |}
\hline
\rowcolor[gray]{0.84}Server & Omschrijving & Nodenaam\\
\hline
\addcolumnfile{nodes.txt}
\hline
\end{tabular}

\section{Artikelen}
\label{artikelen}
\begin{tabular}{| l | l | l |}
\hline
\rowcolor[gray]{0.84}Aantal & Artikel & Omschrijving\\
\hline
\addcolumnfile{artikelen.txt}
\hline
\end{tabular}

\section{Voorbereiding}
\input{voorbereiding.txt}

\section{Randvoorwaarden}
\input{randvoorwaarden.txt}

\chapter{Werkzaamheden}
\input{werkzaamheden.txt}

\section{Aandachtspunten}
\input{aandachtspunten.txt}

\section{Uitvoering}
%\begin{tabular}{| l | p{15cm} |}
\begin{longtable}{| l | p{15cm} |}
\hline
\rowcolor[gray]{0.84}Stap & Omschrijving\\
\hline
\addcolumnfile{uitvoering.txt}
\hline
\end{longtable}

\section{Risico's}
\begin{tabular}{| l | l | l | l | l|}
\hline
\rowcolor[gray]{0.84}Categorie & Risico & Impact & Kans & Maatregelen\\
\hline
\addcolumnfile{risicos.txt}
\hline
\end{tabular}

\section{Planning}
De werkzaamheden vinden plaats op \StartDatum \ tussen \StartTijd \ en \EindTijd.
\section{Go/NoGo}
\input{gonogo.txt}

\chapter{Fallback Scenario}
\label{ch:fallback}

\section{Voorwaarden}
\input{voorwaarden_fallback.txt}

\section{Uitvoering}

\begin{tabular}{| l | p{15cm} |}
\hline
\rowcolor[gray]{0.84}Stap & Omschrijving\\
\hline
\addcolumnfile{fallback_uitvoering.txt}
\hline
\end{tabular}

\chapter{Communicatieplan}
\input{communicatieplan.txt}

\section{Organisatie}
\input{organisatie.txt}

\section{Betrokkenen}
\begin{tabular}{| l | l | l | l | l|}
\hline
\rowcolor[gray]{0.84}Betrokkene & Bedrijf/afdeling & Functie & Telefoon\\
\hline
\addcolumnfile{betrokkenen.txt}
\hline
\end{tabular}

\end{document}
